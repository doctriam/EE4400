\documentclass{article}

\title{Notes: Anwani 10/2018}
\date{2019-02-03}
\author{Kenneth Haynie, Jr}

\begin{document}
    \pagenumbering{arabic}
    \maketitle
    \setcounter{section}{2}
        \section{Spiking Neurons}
            \subsection{Synapse}
                \begin{equation}
                    i(t) = w \alpha(t-t^f)
                \end{equation}
                \begin{itemize}
                    \item $i(t)$: \textbf{analog current signal received by
                    post-synaptic neuron}; linear and time-invariant
                    \item $w\alpha(t)$: transfer function
                    \item $w$: 
                    \begin{itemize}
                        \item synaptic weight representing conductance across
                        a synapse connecting two neurons
                        \item also known as the scaling factor
                        \item varies from synapse to synapse
                        \item can be positve or negative depending on ``\textbf{which
                            the synapse} is said to be excitatory or inhibitory
                        respectively"
                    \end{itemize}
                    \item $\alpha(t)$: post-synaptic current kernel;
                    independent of synapse
                    \item $t^f$: time at which pre-synaptic neuron issues a
                    spike; firing time
                    \item assume synaptic currents are independent of membrane or reversal
                    potential of post-synaptic neuron
                \end{itemize}
                \hfill\break
                \begin{equation}
                    \alpha(t)=[\exp(-t/\tau_1)-\exp(-t/\tau_2)]u(t)
                \end{equation}
                \begin{itemize}
                    \item $\tau_1 > \tau_2$
                    \item $\tau_1$ and $\tau_2$: \textbf{unspecified in document}
                    \item $u(t)$: Heaviside function; models the incoming spike
                \end{itemize}
                \hfill\break
                \begin{equation}
                    c_i(t)=\sum_{f}\alpha(t-t^i_f)=\sum_{i}\delta(t-t^i_f)\ast[e^{-t/\tau_1}-e^{-t/\tau_2}]
                \end{equation}
                \begin{itemize}
                    \item $i$: represents specific synapse
                    \item $t^i_f$: (\textit{most likely}) pre-synaptic firing time of
                    individual spikes across the specified synapse
                    \item $t$: 
                \end{itemize}
\end{document}
